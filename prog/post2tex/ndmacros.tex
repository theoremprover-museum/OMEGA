%***************************************************************************
%   Hier kommen die Makros f"ur die Umgebung ndproof
%***************************************************************************
% 
% Die Umgebung ndproof und ndproofsection erwarten ein Argument, ein Label
% fuer die interne Beweisnummer des Beweises.
% ndproof setzt das label, mit dem dann ueber \ref zugegriffen werden kann.
% Mit ndproofsection kann
% dann ein Teil des Beweises mit dem richtigen Label ausgelagert werden,
% dabei werden alle Zeilen und labels im Kontext dieses Beweises referenziert.
% Der Benutzer muss dann selbst darauf achten, das die labelung konsistent
% bleibt.
% Die Umgebungen ndproof und  ndproofsection
% stellt die Makros \proofheader, \pl, \planline, \ndproofdevider und
% \thmline zur Verf"ugung.
%
% Das Makro \ndproofheader erzeugt eine Kopfzeile mit den Tabellenueber-
% schriften No, S;D, Formula, Reason
% 
% Das Makro \pl erzeugt eine Beweiszeile, die automatisch durchnumeriert
% wird, es hat 4 Argumente:
% <label>, <assump>, <formula>, <reason>; dabei ist <label> ein Name 
% fuer die Zeilenummer, mit dem unter <assump> und <reason> auf die Zeile 
% referenziert werden kann.
% <label> ist dabei eine beliebige Zeichenkette, die keine TeX-Sonder- 
%     Zeichen (\,%,&,$,...), aber auch nicht die Zeichen "," ";" "."
%     und "|" enthalten darf.
% <assump> ist eine durch Kommata getrennte Liste von vorher definierten
%     <labels> 
% <formula> ist die Formel, die in der Zeile stehen soll.
% <reason> ist die Begr"undung fuer die Beweiszeile, sie besteht aus
%     einer Regel, gefolgt duch ein Semikolon ";" gefolgt von einer
%     durch Kommata getrennten Liste von definierten labels.
%     ACHTUNG!! Wenn die Liste von labels leer ist muss immer noch ein
%     Semikolon stehen, wenn nicht: viel Spass beim Suchen in den
%     kryptischen Fehlermelldungen.
%
% \planline hat nur die Argumente 
% <label>, <formula> und erzeugt eine Beweiszeile, wo die
% Assumptionliste nur aus der Zeilennummer besteht und die Begruendung
% aus (PLAN).
% 
% \thmline erzeugt die abschliessende Theoremzeile eines Beweises und
% hat die Argumente <assump>, <formula>, <reason>, da auf diese Zeile
% nicht mehr referenziert werden muss
%
% \ndproofdevider hat ein Argument und zieht einen waagerechten
% Strich durch den Beweis zu Verschoenerungszwecken und schreibt das
% Argument in die Mitte
%
% \pletc erzeugt eine Beweiszeile, in der in jeder Spalte vertikale
% unsoweiter Punkte stehen
%
% \nextlinenumber gibt der folgenden Beweiszeile die Nummer die im Argument
% steht. Ab da wird dann auch automatisch weiternumeriert.
%************************************************************************

\newcounter{plineno}

\def\doref#1{\ref{\ndprooflabel:#1}}

\def\recref#1,#2.{{%
\def\tist{#1}\def\test{#2}%
\ifx\tist\empty{}\else\recorref #1|.\fi%
\ifx\test\empty{}\else,\recref #2.\fi}}

\def\recorref#1|#2.{{%
\def\tist{#1}\def\test{#2}%
\ifx\tist\empty{}\else\doref{#1}\fi%
\ifx\test\empty{}\else\big|\recorref #2.\fi}}

\def\rulerecref#1,#2.{{%
\def\tist{#1}\def\test{#2}%
\ifx\tist\empty{}\else\ \doref{#1}\fi%
\ifx\test\empty{}\else\ \rulerecref #2.\fi}}

\def\ruleref#1;#2.{#1\rulerecref #2,.}

%\def\assref#1;#2.{{\recref #1,.};{\recref #2,.}}
\def\assref#1;#2.{{\recref #1,.}#2}


%\newcommand{\auxpl}[3]%assump,formula,reason
%{\arabic{plineno}. 
%\> \parbox[t]{2.4cm}{\assref #1.}\> $\vdash$ \>
%\begin{minipage}[t]{8.4cm}\raggedright{#2}\end{minipage}
%\> \parbox[t]{2.5cm}{(\ruleref #3.)}\\} % xxx war 3cm 

\newcommand{\auxthmline}[3]%assump,formula,reason
{Thm.
\> \parbox[t]{2.4cm}{\assref #1.}\> $\vdash$ \>
\begin{minipage}[t]{9.4cm}\raggedright{#2}\end{minipage}
\> \parbox[t]{2cm}{(\ruleref #3.)}\\}

\newcommand{\dolabel}[1]{\label{#1}}

\newcommand{\pl}[4]%label,assump,formula,reason
{\refstepcounter{plineno}\dolabel{\ndprooflabel:#1}%
\auxpl{#2}{#3}{#4}}

\newcommand{\nextlinenumber}[1]%lineno
{\ifnum#1>\theplineno%
\setcounter{plineno}{#1}\addtocounter{plineno}{-1}%
\else\typeout{Illegal proofline-number #1 in ND-proof \ndprooflabel}%
\fi}

\newcommand{\planline}[2]%label,formula
{\pl{#1}{#1;}{#2}{PLAN;}}

\newcommand{\openline}[2]%label,assump,formula
{\pl{#1}{#1;}{#2}{Open;}}

\newcommand{\thmline}[3]%assump,formula,reason
{\auxthmline{#1}{#2}{#3}}

\newcommand{\pletc}%
{$\quad\vdots$\>$\quad\vdots$\>\>$\qquad\vdots$\>$\quad\vdots$\\}

\newlength{\boxbreite}\newlength{\strichbreite}
 
\newcommand{\ndproofdevider}[1]%
{\settowidth{\boxbreite}{ #1 }%
\setlength{\strichbreite}{0.5\textwidth}%
\addtolength{\strichbreite}{-0.5\boxbreite}%
\rule[0.8ex]{\strichbreite}{0.3pt} #1 \rule[0.8ex]{\strichbreite}{0.3pt}\\}

\newenvironment{ndproof}[1]%Beweislabel
{\global\def\ndprooflabel{#1}\setcounter{plineno}{0}
\footnotesize\begin{tabbing}%
\hspace*{1.1cm}\=\hspace*{3.0cm}\=\hspace*{0.6cm}%
\=\hspace*{9.5cm}\= \kill}%
{\end{tabbing}}

\newenvironment{ndproofsection}[1]%Beweislabel
{\footnotesize\begin{tabbing}%
\hspace*{1.1cm}\=\hspace*{2.5cm}\=\hspace*{0.6cm}%
\=\hspace*{9.5cm}\= \kill}%
{\end{tabbing}}

\newcommand{\ndproofheader}%
{No \> S;D \> \> Formula \> Reason\\[0.5ex]
\rule{\textwidth}{0.5pt}\\[0.5ex]}

\newcommand{\geru}[2]
{\begin{tabular}{c}{$#1$}\\ \hline{$#2$}\end{tabular}}
\newcommand{\gentzenrule}[3]
{{#1}\geru{#2}{#2}}
\newcommand{\gentzenrulem}[3]
{\gentzenrule{#1}{$#2$}{$#3$}}
\def\premdivider{{\;;\;\;}}

%--------------- Better Style ---------------------

\renewenvironment{ndproof}[1]%Beweislabel
{\global\def\ndprooflabel{#1}\setcounter{plineno}{0}%%
\begin{center}%
\footnotesize\begin{longtable}{lp{2cm}@{$\,\vdash\,$}p{10cm}l}%
%%\footnotesize\begin{longtable}{p{4ex}p{3cm}@{$\,\vdash\,$}p{8cm}l}%
}%
{\end{longtable}\end{center}}
%
\newcommand{\auxpl}[3]%assump,formula,reason
{\arabic{plineno}. &  {\assref #1.} & #2 & {(\ruleref #3.)} \\
} 
%

% \renewcommand{\pl}[4]%label,assump,formula,reason
% {\scriptsize#1.&  \tiny #2 & #3 & \scriptsize (#4)\\
% }

\renewcommand{\pl}[4]%label,assump,formula,reason
{{\scriptsize#1}.& {\tiny #2} & #3 & {\scriptsize (#4)}\\
}

\def\lambdot{\rule{0.6mm}{0.6mm}\hspace{0.4ex}}   % the square dot
\def\lamb#1{\lambda #1\lambdot}
\def\all#1{\forall #1\lambdot}
\def\ex#1{\exists #1\lambdot}

